- Description of next steps
\label{future-work}
Currently AMSI supports only static parallel execution space discretization with non-overlapping process sets. The restriction on non-overlapping process sets is currently being worked on to allow temporary scale-reasignment while a scale is blocked waiting on an associate scale and has no work which can be conducted. However, more dynamic management of the execution space repressents the most significant current goal of the AMSI project. Implementation of this feature -- which will allow scale-tasks to grow and shrink as computational demands associated with the scale change -- is dependent on much if not all of the preceeding functionality (especially load balancing and migration features, as all relevant scale data from a process to be dynamically reassigned scales must be recovered and redistributed throughout the related scale).

Scale-balancing is the balancing of scale-tasks against eachother to minimize rendezvous time and maximize time spent in useful computation. Scale-balancing is a difficult prospect as it involves managing many dynamically-changing scales which themselves are being load-balanced using scale-sensitive load-balancing techinques, and represents a sigificant and promising area of future work. Once the ability to dynamically manage the process set associated with a scale is developed, algorithms must be developed to make the best use of this feature. Developing these algorithms repressents a major - and daunting - area of future work, as there are numerous constraints which must be taken into account in order to optimize such a scheme.

A second micro-scale simulation, where the collagen fibers are embedded in tissue is currently implemented as a standalone simulation. Work to incorporate this into the biotissue example simulation will be conducted when the correct scale-coupling terms have been developed (inclusion in the AMSI system is already accomplished, but the up- and down-scaling operations have not been developed yet so this RVE is not used). \cite{} \cite{} \cite{}

At present AMSI has no meta-model of the computational domain of a problem, or of the relation between the domains of associated scales, simply that they are related. A richer model of the domain relation between problems would likely allow for a wider range of automation in scale-linking and dynamic scale management operations. This will likely not be addressed until a test case where such functionality is needed or beneficial is found.

An article focusing on the implementation and use of the scale-sensitive load-balancing operations in AMSI for the biotissue example problem is being finalized and will be submited for publication soon. Additional articles focused primarily on the usage of the Biotissue example simulation to solve novel problems of interest is also being worked on and should see submission in the coming months.

- Summary of new proposals derived from work
- Follow-on funding

At present there are no current proposals associated with this work for ongoing funding, simply because funding for associated projects (Biotissue and other multi-scale projects) has already been secured. Thus work on AMSI will continue during the work on projects supported by AMSI. Several interesting multi-scale simulations have been suggested, however, particularly a combination of a Monte-Carlo code and a Magnetostatics code by a fellow researcher. At present this idea is in a preliminary stage of development, though using AMSI would be relatively straightforward and achieving a working code (given two working single-scale codes, and the development of physically-sound scale-linking operations) would likely take only several days of programming and debugging.

- Impact on programs

There are several ongoing projects at the Scientific Computation Research Center (SCOREC) on the RPI campus which AMSI will be used with. Particularly SCOREC's ongoing work on incorporating our dynamic mesh operations with the Albany simulation system developed at Sandia National Labs. For multi-scale simulations using the Albany simulation code, we will make use of AMSI's capabilities to be minimally intrusive and only require modifications at those locations in code where multi-scale terms are needed.