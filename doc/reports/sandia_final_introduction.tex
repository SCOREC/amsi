Most physical models and implementations using numerical methods operate at a single scale characteristic to the problem -- the scale at which the preponderance of attributes governing the underlying physics being modeled are measured -- and the scale at which the resolution of the solution is considered comensurate with the desired accuracy of the simulation.

Introduction of influence from scales orders of magnitude removed from the primary scale can take place in several different ways. This influence can be incorporated during the mathematical development of the physical model itself. In this case the multi-scale features are incorporated into the numerical implementation automatically, as a result of their presence in the mathematical model. Multi-scale influences can also be incorporated in a more strictly numerical sense, combining physical models operating at different scales but opera ting on some of the same physical fields and values. 

Numerically multi-scale simulations associate many numerical implementations of single-scale physical models and cause them to interact in a meaningful way by passing key physical terms from some scales and supplying them to others, typically this will involve up-/down-scaling equations to take into account the difference in scale. This is only useful in cases where terms from the various incorporated scales contribute meanignfully to the solution at the primary scale of interest. The domain at the scale of interest (for brevity the engineering scale) is sufficiently large that conducting the entire simulation with a granularity orders of magnitude smaller would make the simulation computationally intractable. 

This restriction requires the development of multi-scale simulation paradigms: algorithms and operations which allow the influence of various far-removed scales to be accurately reflected in the simulation of the engineering scale. While there are several well-established multi-scale paradigms (particularly the heterogenous multi-scale method as well as the concurrent multi-scale method), this is a rapidly advancing field and as new problems are considered additional multi-scale models will undoubtedly be developed. As such developing a set of tools to optimize a single multi-scale paradigm will be less impactful then the development of general tools which can be used themselves to implement new multi-scale paradigms, while supporting extant paradigms as efficiently as possible.