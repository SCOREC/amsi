Currently AMSI provides interfaces for describing and managing the execution of individual scales in a multi-scale simulation, for defining and managing the scale-coupling communication required by such simulations, and for planning and enacting scale-sensitive load balancing operations on individual scales in a multi-scale system.

AMSI operates by maintaining a set of minimal simulation metadata in order to model various quantities of interest during simulation execution. A decentralized approach toward control is taken; dynamic control decisions are made and implemented at runtime during collective operations. In order to avoid the introduction of unnecessary parallel barriers into the code, AMSI control decisions are only made during operations which are already collective over the set of processes effected by a given control decision.

\label{amsi-scales}
During simulation initialization simulation scales are declared and then defined by associated them with process sets, and declaring their scale-linking relations. Process sets are simply mathematical sets of process ranks implemented so as to take advantage of any mathematical conveniences to minimize explicit storage whenever possible. At present process sets must be non-overlapping to take advantage of the simulation control portions of the AMSI libraries, but for the simulation metadata modeling, scale-coupling communication, and multi-scale load balancing operations, the process sets associated with related scales may overlap. Scales are declared to be related if some quantities of interest (such as tensor field values) are transmitted between them for scale-coupling.

\label{amsi-communication}
By using data structures describing the dynamic state of the parallel execution space AMSI can determine which scales will need to interact, and only for these interacting scales are the required internal buffers and scale-coupling structures allocated. This approach is taken in order to reduce unneeded overhead caused by maintaining data structures specific to interacting scale-tasks, which would be the case when assuming that any two scale-tasks may interact. Dynamic redefintion of which scales interact is not currently possible, but will be implemented to support dynamic scale instantiation.

Scale-linking quantities are dynamically registered with the AMSI system by the scale-task which produces the data, along with information about which interacting scale-tasks will receive how much data. This data is stored in AMSI, and when scale-linking communication is to take place, a user-configurable algorithm operating on a mapping of all pertinent scale-linking data is used to provide a full plan of scale-linking communication. This plan is the diseminated (only partially) to those processes which need to know their individual role in the communication. At this point scale-linking communication of any POD datatypes may take place at any time. If the quantity of scale-linking data changes at any point the previous steps must be repeated, but if the quantity remains the same for several instances of inter-scale communication the processes do not need to occur every time. Quantity here is a fairly abstract term, as the discrete units modeled by AMSI metadata structures is arbitrary: they can be any discrete unit of data that can be handled by the underlying communications system (MPI in most cases).

\label{amsi-load-balancing}
Multiscale load balancing processes also make use of the meta-data structures concerning the discretization of the parallel execution space. AMSI currently has separate functionalities for: data removal, data addition with load balancing, and scale-sensitive load-balancing (migration). The removal routine does not automatically call any load balancing or migration routines and should normally be paired with the migration routine. In the current hierarchical mutliscale test problem these routines are called in the previously listed order. Data is first removed which may unbalance the work loads. Next, data is added with load balancing so that differences in workload can be filled by the new data. Finally, scale-senstive load-balancing is used to rectify any remaining unbalance, though as the computational demand (weight) associated with the newly distributed data is typically not known a-priori, some semi-accurate weighting metric should be used to ensure the new data is not treated incorrectly. Currently each of these routines also assumes implicit ordering of data within AMSI since explicit ordering has not yet been implemented. This means that when using the load balancing and migration routines, the user's understanding of the data outside of AMSI needs to be reordered to maintain agreement across scales. This is a side-effect from the use of implicit (user-defined) ordering of scale-coupling data, which has the benefit of not introducing additional memory requirements to explicitly and uniquely number these pieces of data inside of AMSI. Further, AMSI provides the capability to automatically reorder the user data and provide a report on the updated ordering. Still as this may be undesirable in some cases, work is being conducted to provide an alternative option using unique numbering internal to AMSI to avoid effecting user data.

Due to the necessity of retaining inter-scale linkages through the load-balancing processes, the actual data migration undertaken to implement a specific load balancing plan is typically conducted internally by the AMSI system which is provided with buffers of data being load-balanced by each process on the appropriate scale. However, in order to maintain usability in more general situations, it is also possible to simply inform the AMSI system that the current load-balancing plan has already been accomplished by user-implemented data migration. This functionality combined with the ability to use user-designed load-balance planning algorithms allows AMSI to work with entirely external load-balancing libraries and algorithms, allowing scale-sensitive load-balancing to be used even when a specific multi-scale use case falls outside the bounds of capabilities built into the system.
