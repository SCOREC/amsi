\label{sumup}
The development of the Adaptive Multi-scale Simulation Infrastructure has produced an initial set of tools capable of easing the implementation and execution of multi-scale simulations by combining robust legacy codes in a minimally intrusive manner. The AMSI libraries operate independently of the underlying simulation codes, both in terms of language of implementation and the numerics operations undertaken to acheive solution.

This freedom allows AMSI to be used for a wide variety of multiscale cases. However this freedom also means that it is possible to do things poorly, as with any sufficiently low-level system. Thus future developments are likely to be focused on building a hierarchy of abstractions atop the currently implemented low-level systems ease in the use of AMSI for the widest-used subset of the cases AMSI provides support for.

A set of operations for managing adaptive multi-scale simulations has been developed and implmented in AMSI, specifically support for dynamically adding and removing scale-tasks at run time. This support is being built on and enriched to allow this complex operation to happen with minimal programmer intervention, at present the support is low-level and some simulation developer intervention in code to manage the newly-initialized or removed scales is needed. Building on this operation and expanding the metadata model AMSI maintains of the simulation state will allow for the implemtation of fully dynamic simulation scale control: arbitrary initialization and destruction of non-primary scales, and allowing blocked process scales to be reassigned (temporarily or permanently) to higher-priority scales.

Additionally, support for load balancing an individual scale of a multi-scale simulation while maintaining the overall integrity of the multi-scale coupling operations has been developed. This allows more traditional load balancing operations -- vital to optimizing parallel performance and time-to-solution in traditional massively parallel simulations -- to be used on scale codes being used with the AMSI systems. This also represents an important step in the development of general adaptive capabilities for multi-scale simulations, where they greatest challenge is balancing the execution of many interacting (and possibly blocking) simulation scales.
